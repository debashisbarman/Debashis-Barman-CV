\documentclass{resume}

\usepackage[left=0.4 in,top=0.4in,right=0.4 in,bottom=0.4in]{geometry}
\usepackage{hyperref}
\hypersetup{
    colorlinks=true,   
    urlcolor=blue
}
\newcommand{\tab}[1]{\hspace{.2667\textwidth}\rlap{#1}} 
\newcommand{\itab}[1]{\hspace{0em}\rlap{#1}}
\name{Debashis Barman}
\address{6th Cross, AGB Layout, 1st Stage, Jalahalli West, Bengaluru, Karnataka, India, 560057}
\address{+91-88-763-42995 \\ \href{mailto:hello@debashisbarman.in}{hello@debashisbarman.in} \\ \href{debashisbarman.in}{debashisbarman.in}}

\begin{document}

% --------------------------------------------------------
\begin{rSection}{Education}
	
	{\bf Post Graduate Diploma in Computer Science and Applicaton} \hfill {August 2016 - July 2017}
	\\ 
	Institute of Youth Development and Education (IYDE), Tihu, Assam, India
	
	{\bf Bachelor of Technology in Computer Science and Engineering} \hfill {August 2012 - July 2016}
	\\ 
	Assam Don Bosco University (ADBU), Guwahati, Assam, India 
	
\end{rSection}

% --------------------------------------------------------
\begin{rSection}{Skills (Programming \& Software)}
	
	\begin{tabular}{ @{} >{\bfseries}l @{\hspace{6ex}} l }
		Programming Languages & C, C++, Java, Bash, Assembly (i386)                                  \\
		Web Technologies      & HTML, CSS, Javascript / ES6, SASS, Bootstrap, React, Redux, Node.js, \\& Express, REST, GraphQL\\
		Databases             & MySQL, PostgreSQL, MongoDB                                           \\
		Mobile Technologies   & Android, React Native                                                \\
		Tools \& IDE          & Git, NPM, Webpack, Electron, Atom, Nuclide, Android Studio           
	\end{tabular}
	
\end{rSection}

% --------------------------------------------------------
\begin{rSection}{ Research Publication } \itemsep -3pt        
	
	{\textbf{Debashis Barman} and Usha Mary Sharma. ``A study on human activity recognition from video.'' Computing for Sustainable Global Development (INDIACom), 2016 3rd International Conference on. IEEE,} \hfill March 2016. 
	 
\end{rSection}

% --------------------------------------------------------
\begin{rSection}{Academic Projects}
	
	\begin{rSubsection}{Abnormal Human Activity Detection Through Video Surveillance} {August 2015 - June 2016}
		{Supervisor: Prof. Y. Jayanta Singh}{Major Project}
		\item A Linux application to detect crime scene through video surveillance
		\item Developed in C++ and OpenCV
		\item \href{https://github.com/debashisbarman/HumanActivityRecognition}{Github Link}
	\end{rSubsection}
	    
	\begin{rSubsection}{Random Security Code Generation} {January 2015 - July 2015}
		{Supervisor: Prof. Y. Jayanta Singh}{Minor Project}
		\item A Linux application developed in C for generating \textit{Manipulation Detection Code} using Crypographic\\
		Hash Function to ensure data integrity
		\item Recieved Second Prize in Oral Presentation in \textit{A Symposium on E-Vision for Digital India}, ADBU
	\end{rSubsection}
	    
	\begin{rSubsection}{PhD. Scholars' Activity Management} {August 2014 - December 2014}
		{Supervisor: Prof. Y. Jayanta Singh}{Minor Project}
		\item A database based web application to manage the academic activities of the PhD. scholars in ADBU
		\item Developed in HTML, CSS, Javascript, PhP, MySQL
		    
	\end{rSubsection}
	    
	
\end{rSection}

% --------------------------------------------------------
\begin{rSection}{Non Academic Projects}
	
	\begin{rSubsection}{Design of a 32 bit Multi-tasking Kernel} {August 2015 - June 2016}
		{Technology: C, Assembly (i386)}{Hobby Project}
		\item An educational kernel with monolithic design and clean implementation for i386 machines
		\item Entire kernel was written from scratch in C and Assembly (i386)
		\item \href{https://github.com/debashisbarman/minios}{Github Link}
	\end{rSubsection}
	    
	\begin{rSubsection}{Artificially Intelligent Twitter Bot in Node.js} {November 2015}
		{Technology: Node.js, Twitter API}{Hobby Project}
		\item An artificially intelligent Twitter bot developed in Node.js using Twitter API
		\item Featured in \href{https://sdtimes.com/nim-e-book-now-available-google-and-movidius-team-up-on-deep-learning-and-creating-a-twitter-bot-with-node-js-sd-times-news-digest-jan-28-2016/}{SDTimes}, \href{http://www.enhgo.com/code/program-a-twitter-bot/}{Engho} and \href{https://libraries.io/github/debashisbarman/Sofia}{Libraries.io}
		\item Earned appreciation from \textit{Official Node.js Twitter handle} (\href{https://twitter.com/nodejs/status/692736895374102528}{Link}) and a large number of users in \href{https://medium.com/@DebashisBarman/creating-a-twitter-bot-with-node-js-bea760b80bd5}{Medium}
		\item \href{https://gist.github.com/debashisbarman/bffe0f6cd3c0fd2fe40e}{Github Link} and \href{https://twitter.com/digitalsofia}{Live version}
	\end{rSubsection}
	    
	\begin{rSubsection}{Anonymous Online Chat using Web Socket} {August 2015}
		{Technology: Node.js, Socket.io}{Hobby Project}
		\item An online anonymous chatting application using Web Socket
		\item Deployed in \textit{\href{http://adda.herokuapp.com/}{Heroku}, cloud application platform}
	\end{rSubsection}
	
	\begin{rSubsection}{Prajyukttam 2015 Android App} {September 2015}
		{Technology: Android, XML}{Android App}
		\item Official Android App for Prajyukttam 2015, ADBU
		\item Published on \href{https://play.google.com/store/apps/details?id=in.edu.dbu.prajyukttam.prajyukttam2015}{Google Play}
	\end{rSubsection}
	    
\end{rSection}

% --------------------------------------------------------
\begin{rSection}{Internship \& Training} \itemsep -3pt  
	
	{\textbf{Summer Internship} \\Education and Research Network (ERNET), New Delhi } \hfill June 2015 \\
	{\textbf{National Level Workshop cum Summer Internship on Web Application Development} \\Dept. of CSE \& IT, School of Technology, ADBU} \hfill June 2014
	
\end{rSection} 

% --------------------------------------------------------
\begin{rSection}{Workshop \& Invited Talk}
	
	\item \textbf{Workshop on Basics of Kernel Programming} organized by Coding Club, Dept. of CSE \& IT, School of Technology, ADBU
	\item \textbf{Session on Graphics Programming in C} organized by Coding Club, Dept. of CSE \& IT, School of Technology, ADBU
	
\end{rSection}

% --------------------------------------------------------
\begin{rSection}{Professional Experience}
	
	\begin{rSubsection}{Redloop Technologies Pvt. Ltd.} {October 2017 - Present}
		{Full Stack JavaScript Developer}{}
		Currently working as a Full Stack JavaScript Developer in a development studio in Bengaluru that largely focuses on creating Web and Mobile apps using Node.js, React and React Native.
	\end{rSubsection}
	
\end{rSection}

% --------------------------------------------------------
\begin{rSection}{Other Experience}
	
	\begin{rSubsection}{Tutor in Swastyayan, ADBU} {}
		{}{}
		Participated as Tutor in \textit{``Swastyayan : a Commitment''}, voluntary tutoring project implemented by ADBU.
	\end{rSubsection}
	
\end{rSection}

% --------------------------------------------------------
\begin{rSection}{Awards \& Achievements}
	
	\item Awarded 1st Prize, \textbf{Science and Engineering Fair 2015}, Regional Science Center, Guwahati,\\NCSM, Ministry of Culture, Government of India
	\item Secured 2nd Prize, Oral Presentation, \textbf{A Symposia on E-Vision for Digital India}, 2015, ADBU
	\item Secured 2nd Prize, Design Creator, \textbf{Prajyukttam 2015}, ADBU
	\item Secured 2nd Prize, Riddle Snipper, \textbf{Prajyukttam 2015}, ADBU
	\item Semi Finalist, \textbf{CodeUncode 2014 : National Level Secure Programming Competition}, International Council of Electronics Commerce Consultants (EC-Council)
	\item Secured 1st Prize, Creator Pienio, \textbf{Prajyukttam 2014}, ADBU
	\item Secured 2nd Prize, \textbf{Geek's Arena, Prajyukttam 2014}, ADBU
	\item Secured 2nd Prize, \textbf{Web D Page, Prajyukttam 2014}, ADBU
	\item Awarded \textbf{Anundoram Borooah Award}, Government of Assam \item \textbf{Chemistry Olympiad 2008}, Department Of Chemistry, Guwahati University
	    
\end{rSection}

% --------------------------------------------------------
\begin{rSection}{References}
	
	\begin{minipage}[t]{0.5\textwidth}
		\textbf{Dr. Y. Jayanta Singh}\\
		Director, NEILIT, Kolkata\\
		Jadavpur University Campus Area, Jadavpur\\
		Kolkata, West Bengal, 700032\\
		+91-94-353-21669\\
		\href{mailto:yjayanta@gmail.com}{yjayanta@gmail.com}
	\end{minipage}
	\begin{minipage}[t]{0.5\textwidth}
		\textbf{Mr. Dilip Kumar Barman}\\
		Director, ERNET India\\
		Electronics Niketan, 6 CGO Complex\\
		Loadhi Road, New Delhi, 3\\
		+91-11-221-70574\\
		\href{mailto:barman@eis.ernet.in}{barman@eis.ernet.in}
	\end{minipage}
	    
	\begin{rSubsection}{}{}
		{}{}
		\item \small{Last updated on 18 November 2017}
	\end{rSubsection}
	    
\end{rSection}

\end{document}
